\documentclass[10pt,journal,compsoc]{IEEEtran} %draftclsnofoot, 
\usepackage{graphicx}
\graphicspath{ {./_figures/} }
\setlength\fboxsep{1pt}
\setlength\fboxrule{1pt}
\usepackage{multicol}
\bstctlcite{IEEEexample:BSTcontrol}
\usepackage{bigstrut}
\usepackage{tabularx}
\usepackage{tabulary}
\usepackage{booktabs}
\usepackage{amsmath}
\usepackage{balance}
\usepackage{flushend}
\setlength{\parindent}{0em}
\setlength{\parskip}{1em}
\usepackage[caption=false,font=normalsize,labelfont=sf,textfont=sf]{subfig}
\usepackage{hyperref}
\hypersetup{
  colorlinks = false,
  hidelinks = true
	}
\usepackage[english]{babel}
\usepackage{blindtext}
\usepackage{times}
\usepackage{cite}

\begin{document}
  \markboth{BME 560 Medical Imaging: X-ray, CT, and Nuclear Methods. Fall, 
  2014}%
  {BME 560 Medical Imaging: X-ray, CT, and Nuclear Methods. Fall, 2014}
  
  \title{Intensity Modulated Radiation Therapy using Tomotherapy}
  \author{Rahul Krishna, 
  \IEEEauthorblockA{\normalsize {\textit{Dept. of Electrical and Computer 
  Engineering}\\
    North Carolina State University, Email: 
    \href{mailto:rkrish11@ncsu.edu}{{rkrish11@ncsu.edu}}}}}

	\IEEEcompsoctitleabstractindextext{%
  \begin{abstract}
    Tomotherapy is a form of 3D-conformal radiotherapy which involves the 
    delivery of intensity modulated radiation therapy (IMRT) using rotational 
    fan beam in a manner quite similar to that seen in modern CT scanners. The 
    modern tomotherapy systems have a couch and gantry which are in continuous 
    motion, closely resembeling the traditional helical-CT systems. This kind 
    of radiotherapy is hence called helical tomotherapy. Helical tomotherapy 
    delivers IMRT based on the images of the patient in the treatment position, 
    and these systems do this by acquiring CT images of the patient. This paper 
    is a discourse on the current state-of-the-art in the field, foucsing 
    primarily the technique's conceptual working and results of its clinical 
    implementation. 
    
  \end{abstract}
  % IEEEtran.cls defaults to using nonbold math in the Abstract.
  % This preserves the distinction between vectors and scalars. However,
  % if the journal you are submitting to favors bold math in the abstract,
  % then you can use LaTeX's standard command \boldmath at the very start
  % of the abstract to achieve this. Many IEEE journals frown on math
  % in the abstract anyway. In particular, the Computer Society does
  % not want either math or citations to appear in the abstract.
  
  % Note that keywords are not normally used for peerreview papers.
  \begin{IEEEkeywords}
    Tomotherapy, Intensity Modulated Radiation Therapy.
  \end{IEEEkeywords}}
  \maketitle
  
  \section{Introduction}
	\IEEEPARstart{T}{he} past couple of decades have seen tremendous advances in 
	radiation oncology. The advent of faster and smaller computers has helped 
	transform the field of radiation therapy and treatment planning. The term 
	tomotherapy translates to ``slice therapy'', and it is derived from 
	tomography. Tomotherapy is a form of 3D Conformal Radiotherapy (3D-CRT), 
  which aims to conform the spatial distribution of the prescribed dose to a 
  target volume. The volume includes the cancerous cells plus a margin for 
  spatial uncertainties. While doing do it tries to minimize the dose to the
  surrounding normal structures. The tomotherapy system is designed to make use 
  of the tomographic 	reconstruction mathematics for treatment and also for 
  verification. The idea 	was conceived to tackle several major issues that 
  plagued radiation oncology. One of the primary issues was the limitation of 
  target dose that can be 	delivered due to the presence of neighboring 
  sensitive sturctures. One of the 	potential solutions to avoiding this was 
  to make use of multiple radiation 	beams with non-uniform beam intensities. 
  This method allows one to conform 	the radiation to the target region 
  thereby   sparing the sensitive normal 	structures which would otherwise 
  prevent   normal dose delivery to the affected 	region, see Fig. \ref{intro}.
  
  The setup has a 
  linear accelerator mounted on a rotating gantry as seen in CT 
  sacnners. The radiation is delivered to a patient in a helical way, obtained 
  by concurrent gantry rotation and couch/patient travel as shown in Fig. 
  \ref{fig1}. In general, the helical tomotherapy units have significantly 
  different radiation fields when compared to other radiotherapy systems. This 
  difference can be owed to the absence of a flattening filter, a 
  thin target, an electron stopper, a beam hardener and a compact primary 
  collimator. This paper will briefly highlight the 
  radiation characteristics of the helical tomotherapy unit, especially in 
  parts where it differs from standard accelerators.
  
  The rest of this paper is 
  organized as follows. Section \ref{history} talks 
  about the history of tomotherapy. Section \ref{methods} has 3 subsections: 
  \ref{construction} talks about the construction of the tomotherapy unit, 
  \ref{radiation} discusses the radiation characteristics of helical 
  tomotherapy, and \ref{comparisions} brielfy examines the differences between 
  this techique and other conformal therapy techniques. Section \ref{clinical} 
  is a collection of some clinical applications where tomotherapy 
  has, and in some cases has not,  been effective. Finally, section 
  \ref{conclusions} presents concluding 
  remarks on the topic. 
  
   
  \begin{figure}[htbp!]
    \includegraphics[width=\linewidth]{intro}
    \label{intro}
    \caption{An example of treatment planning in a modern 3D-CRT. In this 
    example, a 3D view of the patient, the PTV, spinal cord, and parotid 
    glands, and the 9 intensity modulated beams used to generate the IMRT dose 
    distribution can be seen. \textit{Image courtesy \cite{IMRT}}}
  \end{figure}
  
  \begin{figure*}[t!]
    \centering
  \includegraphics[width=0.89\linewidth]{fig1}
  \caption{A traditional tomotherapy unit, Ref \cite{Mackie1993}. (a) A block 
    diagram showing the ring gantry, and the radiation source}
  \label{fig1}
  
\end{figure*}

\begin{figure}[htbp!]
  \centering
  \includegraphics[width=0.9\linewidth]{fig2}
  \caption{``Running start-stop'' mechanism as proposed in Ref. 
    \cite{Mackie1993}. (a)--(c) }
  \label{fig2}
  
\end{figure}
	%----------------------------------
	% References
  %----------------------------------
  \section{History}
  \label{history}
  Tomotherapy was designed in the University of Wisconsin - Madison and 
  TomoTherapy Inc., Madison, WI. The first prototype investigated in this study 
  was installed at University of Wisconsin Hospital and Clinics in Madison, WI, 
  USA. The first paper on tomotherapy was submitted in 1992, Ref. 
  \cite{Mackie1993}. This paper described most of the details seen in the 
  modern sytems used today. This paper also introducted the idea of using a 
  continuously moving slip-ring gantry. The usage of fan-beam to form a 
  modulating beam, and a temorally modulating collimator assembly (this came 
  to be known as the `Binary Collimator') was also initially proposed in this 
  paper. It also recommended performing 
  intensity modulation without the use of a field flattening filter, thus 
  improving the spectrum across the beam.
  
  Several initial theories proposed in the paper could not be implemented. For 
  instance, to address the target cooling issue that is faced in modern photon 
  beam radiotherapy, the authors proposed engineering a stationary ``hoop'' 
  target with water supplied cooling; this could not be implemented due to the 
  increased collimation complexity. Also, in order to give a steep penumbra to 
  the target region, the authors of the original paper proposed a 
  ``start-stop'' mechanism where the collimator jaws defining the slice width 
  would be 
  close and later on as the patient traverses through the field the opposite 
  would happen. Fig. \ref{fig2}, taken from Ref. \cite{Mackie1993}, shows how 
  this was supposed to happen. But, this idea wasn't implementable either, 
  mainly because the technique required the jaws to be in movement during 
  treatment, optimization of beams near the ‘bow’ and the ‘stern’ of the 
  tumour may limit the amount of dose received.  
  
  For a more comprehensive history the readers might want to refer to 
  \cite{Mackie2006}. The paper talks about the how this idea was 
  conceived and other technological advances led to its current state.
  
  \section{Methods and Materials}
  \label{methods}
  This section briefly discribes various aspects of helical tomography ranging 
  from it's construction and it's radiation characteristics to the differences 
  between it and other IMRT techniques.
  
  \subsection{Construction}
  \label{construction}
  \begin{figure}
    \centering
    \subfloat[][]{
      \includegraphics[width=\linewidth]{fig5}
      \label{fig4a}}\\
    \subfloat[][]{
      \includegraphics[width=\linewidth]{fig4}
      \label{fig4b}}
    \caption{\protect\subref{fig4a} Axial rotation of the fan beam as the 
      patient (represented 
      as a cylinder) traverses laterally (Along the y-axis). 
      \protect\subref{fig4b} A simplified view of the Helical 
      Tomotherapy 
      treatment system (at a gantry angle of 0$^\circ$) in use today. The 
      setup consists of a 6 MV linac to produce an x-ray beam which is 
      collimated down to a fan-beam using 2 moveable jaws (denoted as `y-jaws' 
      in the image). There is also a Binary Multileaf to modulate the beam 
      laterally. \textit{Images courtesy of \cite{Fenwick2004}}}
    \label{fig4}
  \end{figure}
  Figure \ref{fig4} show the basics of dose delivery in the helical tomotherapy 
  system. There is a short in-line 6 MV linear accelerator that rotates on a 
  ring gantry at a source-axis of 85 cms (instead of the usual 100 cms). The 
  patient couch is translated through the gantry bore in the y-direction as 
  shown in \ref{fig4}\protect\subref{fig4a}. The pitch is about 0.2 to 0.5 
  (Pitch is the distance traveled by the couch in one rotation, divided by the 
  field width in the direction of traversal). The pair of y-jaws in 
  \ref{fig4}\protect\subref{fig4b} are used to define the y-width of the beam. 
  These are usually preset to values like 1,2.5, or 5 cm before imaging. 
  Lateral modulation of the beam is achieved using the binary multileaf. This 
  system consists of 64 binary multileaves, each having a width of 6.25 mm 
  projecting onto an isocenter. These leaves transition rapidly between on 
  and off states, the duration of which is made use of to modualate the 
  intensity. Other parameters are tabulated in \ref{tab1}.
  % Table generated by Excel2LaTeX from sheet 'Sheet1'
  \begin{table}[t]
    \centering
    \caption{Helical Tomotherapy Characteristics}
    \begin{tabularx}{1\linewidth}{XX}
      \toprule[0.4mm]
      \textbf{Parameter} & \textbf{Value} \\
      \midrule[0.4mm]
      Gantry Rotation angle & 
      7\textsuperscript{$\circ$} \\
      Projections per revolution & 51 \\
      Rotation velocity & 10 - 60 sec/rotation \\
      Time-per projection & $>$ 196 ms \\
      \bottomrule
    \end{tabularx}%
    \label{tab1}%
  \end{table}%  
  \begin{figure*}[htbp]
    \centering
    \subfloat[][]{
      \includegraphics[width=0.5 \linewidth]{fig6}
      \label{radchara}}
    \subfloat[][]{
      \includegraphics[width=0.5 \linewidth]{fig7}
      \label{radcharb}}
    \caption{\protect\subref{radchara} Shows the photon fluence profile of the 
    tomotherapy treatment beam. Notice the characteristic conical shape of the 
    tomotherapy profile (solid red line) and the flat shape of the conventional 
    linac IMRT system (dashed blue line). 
    \protect\subref{radcharb} Average photon beam spectra of the helical 
    tomography system in two operational modes: imaging (dashed green) and 
    treatment mode (solid red). \textit{Image courtesy of \cite{Jeraj2004}}}
    \label{fig5}
  \end{figure*}
  \subsubsection{A note on Serial Tomotherapy}
  Serial tomography is another variant of tomotherapy that has been in use for 
  sometime now. The first clinical implementation occured as early as 1994. The 
  setup consists of a binary multileaf collimator attached to a conventional 
  linear accelerator. The collimator comprises of two opposing banks of 20 
  tungsten leaves which are driven pneumatically to lie either within or just 
  outside the fan beam radiation. The fan beam itself is modified by arranging 
  the leaves to lie within the radiation field for a variable amount of time.
  
  This can also be achieved by slightly modifying the conventional linear 
  accelerator by adding a binary multileaf collimator to the head of the linac. 
  Axial tomotherapy dose distributions are delivered slice-by-slice, with 
  patients being sequentially and discretely translated through the linac 
  gantry rotational plane between slices, helical distributions are delivered 
  without interruptions; patients are translated smoothly through the bore of 
  the machine as its gantry continuously and synchronously rotates, the 
  therapy equivalent of spiral computed tomography \cite{Fenwick2006}. 
  \subsection{Radiation Characteristics}
  \label{radiation}
  This section aims to briefly highlight the key differences in radiation 
  characteristics of helical tomography and other forms of radiotherapy. One of 
  the key differences is the lack of flattening filter, which makes the dose 
  more uniform are greater depths. As a result of this, the photon fluence 
  profile is shaped differently when compared to a traditional radiotherapy 
  system, as can be seen in Fig. \ref{radchara}. The conical shape of the 
  profile implies that there will be an increased average dose rate and thus 
  reduces the imaging time. 
  
  Figure \ref{radcharb} shows the differences in the photon spectrum between 
  the 
  treatment and imaging modes of helical tomotherapy. Notice that the in the 
  imaging mode the spectrum has a lower energy when compared to treatment mode; 
  this is done so as to obatin better images by improving the contrast. A more 
  comprehensive comparision is documented in Ref. \cite{Jeraj2004}.
  
  \subsection{Comparisions between IMRT techniques}
  \label{comparisions}
    \begin{table*}[t!]
      \centering
      \caption{A comparision between standard and non-standard IMRT delivery 
      systems (Ref. \cite{Fenwick2006})}
      \small{
      \begin{tabulary}{\linewidth}{LLLLLLL}
        \toprule[0.04cm]
        \textbf{System} & \textbf{Gantry} & \textbf{ Beam Geometry} & 
        \textbf{Collimated} & \textbf{Modulated} & \textbf{Imaging} & 
        \textbf{Gating
          Potential} \\\toprule[0.04cm]
        
        Conventional linac plus multileaf & C-Arm & Non-coplanar cone beam & 
        Jaws 
        with conventional
        multileaf & Jaws with conventional
        multileaf & Portal image, fluoroscopy, kvCT, 
        infrared 
        reflectors &
        Breath-hold, 
        beam trigger, multileaf tracking \bigstrut\\\midrule
        Conventional linac minus multileaf & C-Arm & Non-coplanar cone beam & 
        Jaws with/without tertiary attenuator
        & Jaws with/without tertiary attenuator
        & Portal image, fluoroscopy, kvCT, infrared 
        reflectors & Breath-hold,
        beam trigger,
        jaw tracking \bigstrut\\\midrule
        Serial tomotherapy & C-Arm & Fanbeam, coplanar, sequential,
        indexed trajectory & Jaws with binary
        multileaf & Binary multileaf & Portal image, 
        fluoroscopy, kvCT, infrared reflectors & Breath-hold
        during each
        slice \bigstrut\\\midrule
        Helical tomotherapy & Ring  &  Fanbeam,
        coplanar, helical
        trajectory & Jaws with binary
        multileaf & Binary multileaf & MVCT, infrared 
        reflectors & Delivery
        synchronized
        with breathing
        cycle \bigstrut \\\midrule
        Robotic linac & Robotic Arm &  Pencil-beam,
        noncoplanar & Circular collimator & Superposition of pencilbeams by 
        robotic arm & Biplanar radiography, infrared 
        reflectors & Beam 
        trigger,
        robotic
        tracking \bigstrut \\
        \bottomrule
      \end{tabulary}}%
    
      \label{tab:compare}%
    \end{table*}% \be
  Tomotherapy is different from other standard IMRT techiniques such as 
  conformal radiotherapy (CRT) techniques and non-standard IMRT techniques like 
  Cyber knife.   Standard CRT techniques generate highly conformal radiotherapy 
  using radiation fields generated in a few fixed-angle positions. These are 
  generated by linear accelerators (aka. LINAC) which are then modulated by 
  conventional leaf collimators operated in either dynamic mode or the standard 
  step-and-shoot mode. CyberKnife is a modern radiotherapy system developed by 
  Accuray, Inc., primarily for radiosurgery. It makes use of a highly precise 
  robotic arm to delivery the doses. The arm is computer driven to achieve very 
  high precision, it is mounted with a compact 6-MV LINAC whose orientation is 
  determined by 6 different mechanical subsystems. This robotic arm can be 
  moved with a much greater generality than most conventional sytsems. 
  
  Table \ref{tab:compare} highlights the key differences in various systems. 
  For the 
  sake the completeness, the authors of Ref. \cite{Fenwick2006} have 
  dicotomized the conventional system into 2 types, with and without a 
  multileaf collimator, and the tomotherapy into serial and parallel 
  tomotherapy.
  \section{Clinical Applications}
  \label{clinical}
  This section highlights few of the many clinical applications of tomotherapy. 
  \subsection{Total Body Irradiation (TBI)}
  Tomotherapy can be used in establishing total body irradiation (TBI) to gain 
  greater control over the dose distribution and to spare organs that may be at 
  risk. In a study conducted by Ref. \cite{Gruen2013}, with a cohort of 10 
  patients aged 4 - 22 years with acute lymphoblastic or myeloic leukemia, it 
  was shown that, among other benifits, using HT provided excellent 
  conformal lung sparing with mean doses not exceeding 10 Gy. No lung toxicity 
  was observed for over 15 months after follow-up. Also, the overall 
  morbidity   during the treatment period was very low, corresponding to grades 
  1-2 side   effects. Table \ref{tbi_side} from Ref. \cite{Gruen2013} documents 
  other side  effects, notice that none of the side effects for any patient 
  were of grades 3 or 
  4. 
  
  \begin{table}[t]
    \caption{Acute Toxicity (Ref. \cite{Gruen2013})}
    \centering
    \includegraphics[width=\linewidth]{fig8.png}
    \label{tbi_side}
  \end{table}
  
  In general the use of HT was effective in that it provided excellent 
  individual sparing of organs and very little side effects. There are still 
  room for improvements as one of the disadvantages of using Helical Tomography 
  (HT) for TBI is that it 
  can't be used when the body length of the patient exceeds 145 cms. Although 
  this makes it very effective in cases of pediatric oncology, for 
  patients who are taller than this, a better solution needs to be deviced.
  
  \subsection{Whole brain helical Tomotherapy}
  Whole brain helical tomotherapy is a possible treatment option for patients 
  suffering from malignant melanoma with four or more brain metastates. The 
  possiblity of integrated boost with enhanced hippocampal sparing makes the 
  use of HT a very reliable approach.
  
  A study conducted by Ref. \cite{Gondi2010} showed that Helical 
  tomotherapy 
  spared the hippocampus, with a median dose of 5.5 Gy and maximum 
  dose of 12.8 
  Gy. Other linear accelerator based IMRT spared the hippocampus, 
  with a median 
  dose of 7.8 Gy and maximum dose of 15.3 Gy. On a per-fraction 
  basis, mean 
  dose to the hippocampus (normalized to 2-Gy fractions) was reduced 
  by 87\% to 
  0.49 Gy2 using helical tomotherapy and by 81\% to 0.73 Gy2 using 
  LINAC-based 
  IMRT. Target coverage and homogeneity was acceptable with both IMRT 
  modalities, with differences largely attributed to more rapid dose 
  falloff 
  with helical tomotherapy.
  
  Another study was conducted by Ref. \cite{Hauswald2013} to explore 
  and 
  compare the 
  effects of treatment and response to conventional radiotherapy with 
  a 30 Gy 
  dose in 10 fractions and a standard helical tomography with 30 Gy 
  dose, 10 
  fraction, integrated boost of 50 Gy at the regions of metastates 
  with a 
  hippocampal-sparing. This study showed that the even though HT 
  produced 
  excellent homogeneity and conformality with integrated boost, it 
  still has 
  some limitations when compared to classical radiotherapy on account 
  of the 
  fact that all HT plans had a higher dose brain exposure. The 
  effects 
  shouldn't be too much of an issue as the total treatment time is 
  lower, and 
  so too the integral dose, due to the dynamic couch setup and other 
  design 
  aspects of the HT system.
  \subsection{Inoperable Lung Cancer}
  Radiation therapy has become a standard treatment protocol for 
  patients 
  suffering from non-small cell lung cancer and other solitary lung 
  metastatis. 
  The use had been quite controversial. Among other things, the 
  primary 
  concerns were the expansion of the low-dose irradiated lung volume 
  resulting 
  from the coplanar beam delivery and uncertainity in dose 
  distribution due to 
  respiration. 
  
  However, a very recent study published by Ref. \cite{Nagai2014} seem 
  to show 
  that Helical Tomotherapy for lung cancer is both safe and 
  effective. In this 
  study 2 groups of lung tumor patients were treated; one group of 37 
  patients 
  were administered a dose of 48 Gy in 4 fractions and another group 
  of 35 
  patients were administered a dose of 50-60 Gy in 5-8 fractions. The 
  4-fraction group had a median planning target volume (PTV) of 6.9 
  cm\textsuperscript{3} and the 5-8 fraction group had a median PTV 
  of 14 
  cm\textsuperscript{3}. The study reports that only 2 of the 37 and 
  2 out of 
  35 patients in the 4-fraction cohort respectively had Grade-2 
  radiation 
  pneumonitis. No other complications were reported.
  
  \subsection{Colorectal Cancer}
  A study published by Ref. \cite{Yu2013} showed that using 
  tomotherapy 
  could produce acceptable coverage for the target while decreasing 
  the mean dose to sensitive surrounding areas such as bowel, bladder 
  and femur heads. They report that while tomotherapy had a  lower 
  radical 
  dose homogeneity index (rHDI) compared to conventional 3D conformal 
  radiotherapy (3D-CRT) (0.7$\pm$0.01 vs 22.05$\pm$2.76, p$<$0.001),  
  it did significantly reduced the volume of sensitive tissues 
  receiving a large dose; this was measured using $V_{nGy}$, which 
  denotes the percentage of volume receiving a dose greater than n 
  Gy. Figure \ref{fig6}, taken from \cite{Yu2013}, shows the dose 
  reduction 
  acheived by tomotherapy.
  \begin{figure}[t]
		\centering
    \includegraphics[width=\linewidth]{fig9.png}
    \caption{Tomotherapy(dashed blue line) significantly reduce 
    $V_{nGy}$ for the bowel as compared to 3D-CRT (with p<0.001) for 
    all doses except at 5 Gy (p = 0.568), Ref. \cite{Yu2013}.}
  	\label{fig6}
  \end{figure}
  
  \subsection{Arteriovenous Malformations}
  Arteriovenous malformations (AVM) are aberrant fistulous connections between 
  arteries and veins bypassing the normal capillary bed. It occurs in 0.2-0.8\% 
  of the normal population. If untreated cerebral AVMs can cause hemorrhage and 
  subsequent neurological impairments. An attempt was made by Ref. 
  \cite{Krause2013} to 
  validate the performance of modern tomotherapy systems in tackling AVM. 
  
  They 
  reported that there were no significant results to show that tomotherapy 
  outperforms the conventional linac based methods. In fact, as previously 
  reported by many researchers, tomotherapy exposed bigger brain volumes 
  to doses $\ge60\%$ of the prescription dose (average 12.75 Gy), albeit only 
  for a short amount of time. This does, unfortunately, pose risks of 
  permanent postradiosurgery sequelae influenced by the localisation and the 
  volume of brain receiving 12 Gy or more. However, these were only theoretical 
  planning comparison to explore the potential of these methods for small 
  volume radiosurgery. HT radiosurgery might yet be beneficial for AVM with 
  special geometry as well as for patients with brain metastases that benefit 
  from shorter treatment times at the cost of a slightly higher brain exposure.
  
  \section{Conclusion}
  \label{conclusions}
  In conclusion, IMRT with helical tomotherapy is an advanced form of 
  radiotherapy which constantly outperforms several conventional 3D-CRT. It 
  represents one of the most important technical advances in radiation therapy 
  since the advent of medical linear accelerator.
  
  Current helical tomography systems provide excellent target volume 
  conformality while significantly reduction exposure to the surrounding 
  sensitive tissue. This aspect of the tomotherapy system has motivated its use 
  in oncology. Some clincal applications are listed above, this list by no 
  means exhaustive. However, it needs to be mentioned that the results are 
  reported for small cohorts and in many cases reports are only feasibility 
  studies. There is a lot active research in the field and as Ref. 
  \cite{Beavis2004} 
  suggests, helical tomotherapy can be \textit{a} future of IMRT if not 
  \textit{the} future.
  
  \bibliography{Paper}{}
  \bibliographystyle{IEEEtran}
  
\end{document}